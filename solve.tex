\documentclass[14pt]{extarticle}
    \title{Projecting chains of constrained points}
    \author{Morten Silcowitz}
    \date{Feb 2026}

    %\addtolength{\topmargin}{-5cm}
    %\addtolength{\textheight}{5cm}

\usepackage{amsmath}
\usepackage{amssymb}
\usepackage{alltt}
\usepackage{listings}
\newcommand*{\vv}[1]{\mathbf{#1}}
\renewcommand{\familydefault}{\sfdefault}

\usepackage{xcolor}
\definecolor{bg}{RGB}{30,30,30}
\definecolor{fg}{RGB}{220,220,220}
\pagecolor{bg}
\color{fg}


\begin{document}

\maketitle
\thispagestyle{empty}


\section{Formulation}
We're going to solve
\begin{align*}
\min_{\vv x} \quad &  \frac{1}{2} \; \vv{(x-p)}^{\text T} \vv M \vv {(x-p)} \\
\text{s.t} \quad & \vv z = \vv R \vv x \quad \mathrm{ and} \quad \vv z_i^{\text T}\vv z_i = 1
\end{align*}
where
\begin{align*}
n \in \mathbb N &\quad\text{is the number of particles between links}\\
m \in \mathbb N &\quad\text{is the } n-1 \text{ of links between particles}\\
 \vv x \in \mathbb R^{\text 3n } &\quad\text{is the solution vector,} \\
 \vv p \in \mathbb R^{\mathrm{3n}} &\quad\text{is the vector of target/free fall positions,}\\
 \vv z \in \mathbb R^{\mathrm{3m}} &\quad\text{is the vector of difference
 vectors in the chain,} \\
 \vv z_i \in \mathbb R^{\mathrm{3}} &\quad\text{is the $i$th 3-vector in $\vv z$,} \\
\vv R \in \mathbb R^{\mathrm{3m\times3n}} &\quad\text{is the per vector
finite-difference matrix}\\&\quad\text{so }(\vv R \vv x)_i = \vv x_{i} - \vv
x_{i+1} \\
 \vv M \in \mathbb R^{\mathrm{3n\times3n}} &\quad\text{is the mass matrix,
 simply having}\\ &\quad\text{the mass (or weight) of each
 particle}\\&\quad\text{in each 3x3 block diagonal} \\
\end{align*}
We want to be working only on the difference vectors in $\vv z$. To do that, we
skip the details and jump directly to the minimizer for $\vv x$, treating $\vv
z$ as a constant for now:
\begin{align*}
\vv x = \vv p - \vv M^{-1}\vv R^{\mathrm T} \vv S^{-1}(\vv R\vv p - \vv z)\\
\end{align*}
where $\vv S = \vv R \vv M^{-1} \vv R^{T} \in \mathbb R^{3m \times 3m}$ is a
block tridiagonal matix. Inserting this into the main problem eliminates $
\mathbf z = \mathbf R \mathbf x $ and we are left with


\begin{align*}
\min_{\vv x} \quad &  \frac{1}{2} \; \vv{(z-Rp)}^{\text T} \vv S^{-1} \vv
{(z-Rp)} \\
\text{s.t} \quad & \vv z_i^{\text T}\vv z_i = 1 \quad i=1\dots m
\end{align*}
In this reformulated problem we must find a $\vv z$ that consist of unit length
3-vectors that minimize the distance to $\vv{Rp}$ under the $\vv S^{-1}$ norm.
It worth noting that it is trivial to project $\vv z$ to satisfy the unit length
constraints, and likewise it is trivial to obtain $\vv x$ given any $\vv z$.
This means that whenever we take a newton step on $\vv z$ we can trivially reign
it in to its feasible solution, and in effect only changing the directions of
each $\vv z_i$ vector, never their length.

\section{Algorithm}
Here we'll outline the algorithm in full before explaining each step in detail:
\begin{center}
\begin{lstlisting}[mathescape=true,numbers=right]
$\vv L$ = cholesky($\vv{S}$)
$\vv z$ = $\vv{Rx}$
while true do
    $\vv Q$,$\vv z$ = normalize($\vv z$)
    $\vv y$ = $\vv{Rp-z}$
    $\vv \lambda$ = solve($\vv{QSQ^{T}}$, $\vv{Qy}$)
    $\vv{b_z}$ = solve($\vv L$, $\vv{y}$)
    $\vv{b_\lambda}$ = $\vv{L^{T}Q^{T}\lambda} $
    if $\|\vv{bz-bl}\|^{2}$ < $\epsilon$ then
        $\vv s$ = solve($\vv{L^{T}}$, $\vv{b_z}$)
        $\vv x$ = $\vv{p-M^{-1}R^{T}s}$
        return $\vv x$
    $\vv D$ = diagonal(max($0$,$\vv \lambda$))
    $\vv{\Delta z}$ = $\vv L$ solve($\vv{I+L^{T}DL}$, $\vv{b_z-b_\lambda}$)
    $\vv z$ = $\vv{z+\Delta z}$
\end{lstlisting}
\end{center}
Each iteration the algorithm does these main steps:
\begin{enumerate}
    \item Normalize the 3-vectors in $\vv z$
    \item Compute the $n-1$ Lagrange multipliers of the system at $\vv z$
    \item Check the solution residual, if below threshold compute and return $\vv x$
    \item Compute the Newton step on $\vv z$ and go to step 1
\end{enumerate}
Note the following:
\begin{itemize}
    \item we compute the cholesky decomposition $\vv{LL^{T}=S}$ to keep the
    systems we need to solve sparse and symmetric. See more details bellow.
    Computing cholesky for a block-tridiagonal SPD matrix is simple and fast,
    and if $\vv M$ is fixed it can be done off-line/once.
    \item every \texttt{solve()} is at least a block tri-diagonal system (or
    simpler), which are fast and easy to solve
    \item to compute the Lagrange multipliers we build the Jacobian matrix $\vv
    Q \in \mathbb R^{3(n-1)\times(n-1)}$, which is simply the $\vv z_i$ vectors
    vertically stacked
    \item the matrix $\vv D$ is the contribution to the Hessian stemming from
    the non-linear unit length constraints. We clamp it to be positive to keep
    the Hessian PD. This is a hack but seems to play out well in practice
\end{itemize}

\subsection{Lagrange multipliers}
\subsection{Hessian system}


\subsection{Lagrange multipliers}
To derive the expression for \( \vv x \) as a function of \( \vv z \) using
Lagrange multipliers, we start by considering the constrained optimization
problem:

\[
\min_{\vv x} \quad \frac{1}{2} \; \vv{(x-p)}^{\text T} \vv M \vv {(x-p)}
\]

subject to the constraints:

\[
\vv z = \vv R \vv x \quad \text{and} \quad \vv z_i^{\text T}\vv z_i = 1
\]

We introduce Lagrange multipliers \( \vv \lambda \) for the constraints \( \vv z = \vv R \vv x \). The Lagrangian is given by:

\[
\mathcal{L}(\vv x, \vv \lambda) = \frac{1}{2} \; \vv{(x-p)}^{\text T} \vv M \vv {(x-p)} + \vv \lambda^{\text T} (\vv z - \vv R \vv x)
\]

Taking the derivative of the Lagrangian with respect to \( \vv x \) and setting it to zero gives:

\[
\vv M (\vv x - \vv p) - \vv R^{\text T} \vv \lambda = 0
\]

Solving for \( \vv x \), we have:

\[
\vv x = \vv p + \vv M^{-1} \vv R^{\text T} \vv \lambda
\]

Substituting the constraint \( \vv z = \vv R \vv x \) into the expression for \( \vv x \), we get:

\[
\vv z = \vv R (\vv p + \vv M^{-1} \vv R^{\text T} \vv \lambda)
\]

\[
\vv z = \vv R \vv p + \vv R \vv M^{-1} \vv R^{\text T} \vv \lambda
\]

Rearranging gives:

\[
\vv R \vv M^{-1} \vv R^{\text T} \vv \lambda = \vv z - \vv R \vv p
\]

Let \( \vv S = \vv R \vv M^{-1} \vv R^{\text T} \), then:

\[
\vv S \vv \lambda = \vv z - \vv R \vv p
\]

Solving for \( \vv \lambda \):

\[
\vv \lambda = \vv S^{-1} (\vv z - \vv R \vv p)
\]

Substituting back into the expression for \( \vv x \):

\[
\vv x = \vv p + \vv M^{-1} \vv R^{\text T} \vv S^{-1} (\vv z - \vv R \vv p)
\]

Simplifying, we obtain:

\[
\vv x = \vv p - \vv M^{-1} \vv R^{\text T} \vv S^{-1} (\vv R \vv p - \vv z)
\]


\end{document}

